% Chapter 4

\chapter{Introdução} % Main chapter title
\label{chap:Chapter4} % For referencing the chapter elsewhere, use \ref{chap:Chapter4} 
Este capítulo apresenta a organização responsável pela ideia do tema a ser desenvolvido, o contexto do seu surgimento e os problemas que a originaram. São abordados os objetivos, o planeamento e as considerações éticas tidas em conta para executar a proposta. Por último é descrita a estrutura deste documento de forma a facilitar a leitura do mesmo.
%----------------------------------------------------------------------------------------
\section{Organização e Contexto}
A Critical Techworks (CTW) nasceu de um empreendimento conjunto entre a Critical Software e o grupo BMW (Bayerische Motoren Werke). Estabelecida em 2018 esta empresa desenvolve software exclusivamente para os produtos da BMW. A sua missão é a de "transformar a forma como o mundo se move" focando-se para isso na transformação digital com as melhores práticas do mercado.

A organização estrutura-se em equipas ágeis multidisciplinares, agrupadas em unidades e domínios tecnológicos, o que promove autonomia, ciclos de desenvolvimento rápidos e colaboração contínua. Esta abordagem permite alinhar o desenvolvimento de software com necessidades reais dos veículos e serviços digitais da BMW.

O presente relatório foi desenvolvido no âmbito da unidade curricular de Preparação para Dissertação (PREPD) do Instituto Superior de Engenharia do Porto (ISEP), com o intuito de demonstrar o estudo de viabilidade de uma solução para o problema em causa.

\section{Problema}
 A análise e resolução de incidentes está no dia a dia das equipas que lidam com projetos potencialmente críticos. Um incidente pode ter por origem duas principais fontes: automático, através de métricas definidas pelas equipas; e manuais, criados por clientes das aplicações desenvolvidas. Um registo de incidente (ticket) é um documento, normalmente gerido por um sistema externo (como \textit{ServiceNow} ou \textit{Jira Service Management}), onde é possível visualizar detalhes do incidente, tais como a sua descrição e prioridade, e onde um elemento da equipa pode descrever os passos de resolução efetuados.
 
 Equipas destes sistemas criticos lidam com muitos incidentes diariamente e muitas vezes fora de horários.
 - Quantidade de tickets pode ser demasiada
 - Analise pode ser descuidada, principalmente para o caso de acontecer de madrugada

\section{Objetivos}

\section{Planeamento}
Gant chart...

\section{Considerações Éticas}

\section{Estrutura do Documento}
