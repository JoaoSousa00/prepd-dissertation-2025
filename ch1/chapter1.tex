% Chapter 4

\chapter{Introdução} % Main chapter title
\label{chap:Chapter4} % For referencing the chapter elsewhere, use \ref{chap:Chapter4} 
Este capítulo apresenta a organização responsável pela ideia do tema a ser desenvolvido, o contexto do seu surgimento e os problemas que a originaram. São abordados os objetivos, o planeamento e as considerações éticas tidas em conta para executar a proposta. Por último é descrita a estrutura deste documento de forma a facilitar a leitura do mesmo.
%----------------------------------------------------------------------------------------
\section{Organização e Contexto}
A Critical Techworks (CTW) nasceu de um empreendimento conjunto entre a Critical Software e o grupo BMW (Bayerische Motoren Werke). Estabelecida em 2018 esta empresa desenvolve software exclusivamente para os produtos da BMW. A sua missão é a de "transformar a forma como o mundo se move" focando-se para isso na transformação digital com as melhores práticas do mercado.

A organização estrutura-se em equipas ágeis multidisciplinares, agrupadas em unidades e domínios tecnológicos, o que promove autonomia, ciclos de desenvolvimento rápidos e colaboração contínua. Esta abordagem permite alinhar o desenvolvimento de software com necessidades reais dos veículos e serviços digitais da BMW.

O presente relatório foi desenvolvido no âmbito da unidade curricular de Preparação para Dissertação (PREPD) do Instituto Superior de Engenharia do Porto (ISEP), com o intuito de demonstrar o estudo de viabilidade de uma solução para o problema em causa.

\section{Problema}
A análise e resolução de incidentes está no dia a dia das equipas que lidam com projetos potencialmente \textbf{críticos}. Um incidente é um \textbf{evento} que provoca \textbf{interrupção} ou \textbf{degradação} num qualquer serviço, e podem \textbf{comprometer} significativamente a disponibilidade dos serviços e resultar em perdas financeiras substanciais. [How to Mitigate the Incident? An Effective Troubleshooting Guide Recommendation Technique for Online Service Systems] Estes incidentes podem ter por origem duas principais fontes: \textbf{automático}, através de métricas definidas pelas equipas; e \textbf{manuais}, criados por clientes das aplicações desenvolvidas. Um \textbf{registo de incidente} (\textit{ticket}) é um documento, normalmente gerido por um sistema externo (como \textit{ServiceNow} ou \textit{Jira Service Management}), onde é possível visualizar detalhes do incidente, tais como a sua descrição e prioridade, e onde um elemento da equipa pode descrever os passos de resolução efetuados.

Na área de TI (Tecnologias de Informação), as equipas podem manter vários serviços em simultâneo e por essa razão os engenheiros de plantão (membros das equipas responsáveis por manter os serviços estáveis \textbf{24x7}) podem ficar \textbf{sobrecarregados} de incidentes, fazendo com que tenham que lidar com eventos provenientes de vários serviços paralelamente. Numa situação como esta, os engenheiros podem ter um sentimento de \textbf{confusão}, misturando temas e errando nas suas análises, o que pode causar piores resoluções de incidentes e o aumento do seu tempo de mitigação (\textit{\textbf{TTM}}). Outro cenário frequente para os \textit{\textbf{OCEs}} (\textit{On-Call Engineers}) envolve incidentes que ocorrem fora do horário de expediente, especialmente durante a madrugada, quando o engenheiro de prevenção pode não estar na sua plena capacidade de raciocínio e operacional.

Foi também referido anteriormente os incidente com origem manual. Estes, são criados por clientes finais e/ou equipas de testes e podem conter mais ou menos informação. Muitas vezes, os \textit{OCEs} são \textbf{obrigados} a fazer uma \textbf{extensa análise} para compreender o conteúdo do incidente e a resolução mais adaptada a aplicar.

Uma interpretação ou resolução inadequada das ocorrências pode \textbf{reduzir a disponibilidade} dos serviços, o que, por sua vez, pode gerar \textbf{prejuízos significativos} para a organização, seja pela perda de receitas, seja pelo impacto negativo na experiência do utilizador final.

\section{Objetivos}

\section{Planeamento}
Gant chart...

\section{Considerações Éticas}

\section{Estrutura do Documento}
